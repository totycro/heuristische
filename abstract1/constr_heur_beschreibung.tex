\section{Construction heuristics}

We implemented two different construction heuristics:

\subsection{Greedy ants}
This heuristic is based on the greedy algorithm we implemented in exercise 1: For every round, a city is chosen for game selection.
Cities with few possible games are preferred.
For the chosen city, all possible games are evaluated according to this objective function:
$$ p_{ij} = \alpha * \eta_{ij} + \tau_{ij}  $$
Among these, the minimum is selected.

The local information $\eta_{ij}$ (distance from the current location to all other cities) is scaled into the range of $[0,1]$ by dividing by the maximum distance in order to be able to make it compatible with the pheromone values.
The factor $\alpha$ is used to reduce the influence of local information on the decision over time.

The pheromone values $\tau_{ij}$ are initialized by 1.0.
Higher ranked ants (ants that are the most successful) are able to apply more
pheromones and in our case three ants are allowed to apply them.

\subsection{Randomized ants}

This construction approach is more dynamic and also general than the first one.
Instead of leveraging local information, the decisions are solely based upon the distribution of pheromones. 
The algorithm can therefore select solution attributes, that seem to be bad locally and those will be rewarded, if it leads to a good overall solution, which is not possible with greedy.

Since the search is purely random at the beginning, it takes far more iterations/ants than the greedy approach to find reasonable solutions, but this can be attributed to the fact that a far greater portion of the search space is considered. 

In order to maximize the number of solutions that can be found, we also allow invalid solutions with respect to the repeaters- and number of home- and away-games-constraints in this approach.
Those are punished by a fixed fine per violation so that nearly valid solutions are competitive against bad valid solutions, and solutions with an unreasonable number of violations are discarded.

