\section{Greedy construction heuristic}

The construction heuristic used in this task works in a greedy manner, i.e. it comes to a decision only considering local information and choosing the option that seems optimal with respect to this data. These decisions are never revoked and in general lead to a non-optimal solution, since it's never possible to accept temporary drawbacks that would simplify the further steps and enable better solutions. Furthermore, in order to support GRASP, it has to be randomized.

Our intention was therefore to have an heuristic that can generate very different solutions so that GRASP would not be too much restricted in its discovery of the search space. The basic procedure works as follows:

\begin{lstlisting}
PROCEDURE greedy:
	for each round:
		while not round complete:
			city = choose a city that doesn't play yet
			games = possible games according to constraints that include city
			addGame( min_travel_distance(games) )
\end{lstlisting}

So for every city that is randomly picked, the best game with respect to this part of the whole solution is chosen. The perturbation of the order in which cities are picked leads to very different solution, where the first picks of a round determine most of the rest of it, which is then mostly determined by constraints.

One major drawback of this approach are dead ends which it is unable to avoid. Since global information isn't considered (as it is the nature of a greedy procedure), it can easily happen that if not many games are possible for cities in a future round, all of them will be rendered impossible by a current decision.

Keeping this in mind, our tests with common input instances showed that this procedure produces valid solutions after an acceptable amount of time, if the constraints are not set too tightly. 
