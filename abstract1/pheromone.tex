\section{Pheromone model}

For the TTP, there are several reasonable ways in which pheromones can be distributed to attributes of solutions. We identified several basic alternatives:

\begin{description}
	\item[$n\times n$-edge-matrix.] 
		The entries $(i,j)$ of this matrix would encode the desirability for a game against $j$ after $i$. This is however very limited, i.e. it does not include information about the location of the game, which is most crucial for the value of objective function.
	\item[$2n\times 2n$-edge-matrix.] 
		This can be interpreted as an extension of the first variant, which distinguishes between home- and away-games, i.e. the range $[n+1, 2n]$ represents games against this city abroad. 
		Here, the problem still occurs that an away-game after a home-game can mean very different travelling costs depending on the location of the city with respect to the particular opponent.
		
	\item[Edge-matrix per city.]
		To account for the different locations of the cities, one could consider the travelling order for every city as a distinct problem, therefore keeping a pheromone matrix for every city.
		The information can be expected to be more precise with at the expense of far higher construction costs.
		This approach can be combined with either one of the first two approaches.

	\item[Position-matrix.]
		Instead of saving the edges, one could also save the absolute position of cities in the solution. Since the costs of a solution are composed of the single moves from one location to the next one, this alternative is not advisable. 
\end{description}

In our implementation, we decided to use one $2n\times 2n$-edge-matrix in order to profit from synergies of applying the results of the tours for each city to one piece of data.
The idea is that if many cities e.g. play an away-game against $b$ after they played against $a$, the edge between those two is very likely to be a good solution attribute, and should therefore be used by as many cities as possible. 
This would not be possible in the same way if a matrix per city was used.

	

